Minesweeper is a game where a set number of mines are secretly scattered around the board. The player’s goal is to reveal or clear out all the cells of the boards except for the cells where the mines are. To accomplish this, the player picks a cell at a time and determines whether that cell has a mine. If the player thinks the chosen cell has a mine underneath, then the player flags it (or just leaves it alone). If the player thinks the cell doesn’t have a mine, then the player can clear it. To help the player solve the puzzle, cells that don’t have a mine have numbers that tell the player how many mines surround that particular cell. The player automatically loses if they clear a cell containing a mine.

This game was chosen for the project because I am familiar with the rules of the game and I thought it would be simple enough to implement in code. Not to mention, I thought it would be fun to implement a game board that essentially reveals what it is hidden underneath, be it a value or a mine. This feature of revealing what is inside of a cell is important because it provided a way to apply the concepts learned.

For Project 2, this game was selected again, in the interest of time. On top of that, the board also presents means to transition the code with each cell now acting as objects and having its own properties. ~\newline
 